%% vim: set sts=4 et :

\chapter{Usage}
\label{ch:usage}

EOS has been authored with two use cases in mind.\\

The first such use case is the evaluation of observables and further theoretical
quantities in the field of flavor physics. EOS aspires to produce such evaluations
in the course of theory estimates of publication quality.

The second use case is the inference of parameters from experimental observations.
For this task, EOS defaults to the Bayesian framework of parameter inference.\\

In the remainder of this chapter, we document the usage of the existing
EOS clients and scripts, in order to carry out tasks corresponding to the above use cases.
We assume further that only the built-in observables, physics models and experimental
constraints are used.

\section{Evaluating Observables using \client{eos-evaluate}}
\label{sec:usage:eos-evaluate}


Observables can be evaluated using the \client{eos-evaluate} client. It accepts
the following command line arguments:
\begin{itemize}
    \item[] \texttt{--kinematics NAME VALUE}\\[\medskipamount]
        Within the scope of the next observable, declare a new kinematic
        variable with name \texttt{NAME} and numerical value \texttt{VALUE}.

    \item[] \texttt{--range NAME MIN MAX POINTS}\\[\medskipamount]
        Within the scope of the next observable, declare a new kinematic
        variable with name \texttt{NAME}. Subdivide the interval [MIN, MAX]
        in POINTS subintervals, and evaluate the observable at each subinterval
        boundary.\\

        \emph{Note}: More than one \texttt{--range} command can be issued per
        observable, but only one \texttt{--range} command per kinematic variable.

    \item[] \texttt{--observable NAME}\\[\medskipamount]
        Add a new observable with name \texttt{NAME} to the list of observables
        that shall be evaluated. All previously issued \texttt{--kinematics}
        and \texttt{--range} arguments apply, and will be used by the new obervable.
        The kinematics will be reset (i.e., all kinematic variables will be removed)
        in anticipation of the next \texttt{--observable} argument.
\end{itemize}
The output of calls to \client{eos-evaluate} is structured as follows:
\begin{itemize}
    \item The first row names the observable at hand, as well as all active options.
    \item The second row contains column headers in the order:
        \begin{itemize}
            \item kinematics variables,
            \item the upper and lower uncertainty estimates for each individual uncertainty budget,
            \item the total upper and lower uncertainty estimates.
        \end{itemize}
    \item The third row contains the result as described by the above columns. In addition, at
        the end of the row the relative total uncertainties are given in parantheses.
\end{itemize}
The above structure repeats itself for every observable, as well as for each variation point
of the kinematic variables as described by occuring \texttt{--range} arguments.\\

As an example, we turn to the evaluation of the $q^2$-integrated branching ratio
$\mathcal{B}(\bar{B}^0\to \pi^+\ell^-\bar\nu_\ell)$.
For this example, let us use the integration range
\begin{equation*}
    0\,\GeV^2 \leq q^2 \leq 12\,\GeV^2\,.
\end{equation*}
Further, let us use the BCL2008 \cite{Bourrely:2008za} parametrization of the $\bar{B}\to \pi$ form factor,
as well as the Wolfenstein parametrization of the CKM matrix. The latter is achieved
by choosing the physics model 'SM'. By default, EOS uses the most recent results of
the UTfit collaboration's fit of the CKM Wolfenstein parameter to data on tree-level decays.
In this example, we will evaluate the observable, and estimate parametric uncertainties
based on the naive expectation of Gaussian uncertainty propagation. We will classify two
budgets of parametric uncertainties: one for uncertainties pertaining to the form factors
(labelled 'FF'), and one for uncertainties pertaining to the CKM matrix elements (labelled 'CKM').

Our intentations translate to the following call to \client{eos-evaluate}:
\commandlineexample{examples/evaluate-btopilnu-integrated}

\section{Producing Random Parameter Samples}
\label{sec:usage:eos-sample-mcmc+pmc}

Both use cases, observable evaluation and Bayesian parameter inference, make use of
random samples of some \gls{PDF} $P(\vec\theta)$. These random
samples can be produced from Markov chains, using the Metropolis-Hastings algorithm,
by calls to the \client{eos-sample-mcmc} client. In second step, refined samples,
or samples for a very complicated setup, are obtained from an algorithm described
in Ref.~\cite{Beaujean:2013}. This algorithm uses an adaptive importance sampling
called Population Monte Carlo (PMC), implemented within the client \client{eos-sample-pmc}.

The \client{eos-sample-mcmc} clients accepts the following command-line arguments:
\begin{itemize}
    \item[] \texttt{--scan NAME --prior flat MIN MAX}\\[-3\medskipamount]
    \item[] \texttt{--scan NAME [ABSMIN ABSMAX] --prior gaussian MIN CENTRAL MAX}\\[-3\medskipamount]
    \item[] \texttt{--nuisance [...]}\\[\medskipamount]
        These arguments add a parameter to the statistical analysis, with
        either a flat or a gaussian prior. If \texttt{ABSMIN} and
        \texttt{ABSMAX} are specified, the prior will be cropped to this
        absolute interval.  The \texttt{--scan} and \texttt{--nuisance}
        arguments work identically, with one exception: \texttt{--nuisance}
        declares the associated parameter as a nuisance parameter, which is
        flagged in the HDF5 output. The sampling algorithm \emph{does not}
        treat nuisance parameters differently than scan parameter.

    \item[] \texttt{--seed [time|VALUE]}\\[\medskipamount]
        This argument sets the seed value for the \gls{RNG}. Setting the
        seed to a fixed numerical \texttt{VALUE} ensures reproducibility of the results. This
        is important for publication-quality usage of the client. If \texttt{time} is
        specified, the \gls{RNG} is seeded with an interger value based on the current time.

    \item[] \texttt{--prerun-min VALUE}\\[\medskipamount]
        For the prerun phase of the sampling algorithm, set the minimum number of
        steps to \texttt{VALUE}.

    \item[] \texttt{--prerun-max}\\[\medskipamount]
        For the prerun phase of the sampling algorithm, set the maximum number of
        steps to \texttt{VALUE}.

    \item[] \texttt{--prerun-update}\\[\medskipamount]
        For the prerun phase of the sampling algorithm, force an adaptation of the
        Markov chain's proposal function to its environment after every \texttt{VALUE}
        steps.

    \item[] \texttt{--store-prerun [0|1]}\\[\medskipamount]
        Either disable or enable storing of the prerun samples to the output file.\\

        \emph{Note}: Samples from the prerun should only be used for diagnostic purpose.

    \item[] \texttt{--output FILENAME}\\[\medskipamount]
        Use the file \texttt{FILENAME} to store the output, using the HDF5 file format.
\end{itemize}

The \client{eos-sample-pmc} clients accepts the following command-line arguments:
\begin{itemize}
    \item[] \texttt{--scan NAME --prior flat MIN MAX}\\[-3\medskipamount]
    \item[] \texttt{--scan NAME [ABSMIN ABSMAX] --prior gaussian MIN CENTRAL MAX}\\[-3\medskipamount]
    \item[] \texttt{--nuisance [...]}\\[\medskipamount]
        These arguments add a parameter to the statistical analysis, with
        either a flat or a gaussian prior. If \texttt{ABSMIN} and
        \texttt{ABSMAX} are specified, the prior will be cropped to this
        absolute interval.  The \texttt{--scan} and \texttt{--nuisance}
        arguments work identically, with one exception: \texttt{--nuisance}
        declares the associated parameter as a nuisance parameter, which is
        flagged in the HDF5 output. The sampling algorithm \emph{does not}
        treat nuisance parameters differently than scan parameter.

    \item[] \texttt{--seed [time|VALUE]}\\[\medskipamount]
        This argument sets the seed value for the \gls{RNG}. Setting the
        seed to a fixed numerical \texttt{VALUE} ensures reproducibility of the results. This
        is important for publication-quality usage of the client. If \texttt{time} is
        specified, the \gls{RNG} is seeded with an interger value based on the current time.

    \item[] \texttt{--pmc-initialize-from-file HDF5FILE}\\[\medskipamount]
        Use the samples from a MCMC HDF5 output file \texttt{HDF5FILE} as
        generated with \client{eos-sample-mcmc}, in order to initialize the
        mixture density of the initial PMC step.

    \item[] \texttt{--pmc-group-by-r-value R}\\[\medskipamount]
        When forming groups of MCs from the initialization file, only add
        a chain to an existing group if the chain's $R$-value is less than \texttt{R};
        create a new group otherwise.

    \item[] \texttt{--hc-target-ncomponents N}\\[\medskipamount]
        When creating mixture components, create \texttt{N} components per existing
        MC group.

    \item[] \texttt{--hc-patch-length LENGTH}\\[\medskipamount]
        When clustering a group's MCs onto the mixture components, cut the
        chains into patches of \texttt{LENGTH} samples each.

    \item[] \texttt{--hc-skip-initial FRACTION}\\[\medskipamount]
        Skip the first \texttt{FRACTION} of all MCMC samples in the clustering step.

        \emph{Note}: \texttt{FRACTION} must be a floating point number between $0$ and $1$.

    \item[] \texttt{--pmc-samples-per-component N}\\[\medskipamount]
        Set the number \texttt{N} of samples that will be drawn per component and
        update step of the PMC run.

    \item[] \texttt{--pmc-final-samples N}\\[\medskipamount]
        Set the number \texttt{N} of samples that will be drawn for the final step,
        i.e.: after the PMC updates have converged.

    \item[] \texttt{--pmc-ignore-ess [0|1]}\\[\medskipamount]
        Set whether convergence of the PMC updates shall be determined from the
        effective sample size (ESS) \emph{in addition} to the perplexity.

        \emph{Default}: Use the ESS.

    \item[] \texttt{--pmc-relative-std-deviation-over-last-step STD STEPS}\\[\medskipamount]
        If both perplexity and ESS have a standard deviation less than \texttt{STD} over the
        last \texttt{STEPS} updates, declare convergence.

    \item[] \texttt{--output FILENAME}\\[\medskipamount]
        Use the file \texttt{FILENAME} to store the output, using the HDF5 file format.
\end{itemize}

As an example, we define the a-priori \gls{PDF} for a study of the decay $\bar{B}\to \pi^+\ell^-\bar\nu_\ell$.
For the CKM Wolfenstein parameters, we use
\begin{equation*}
\begin{aligned}
    \lambda    & = 0.22535 \pm 0.00065\,,  &
    A          & = 0.807 \pm 0.020\,,      \\
    \bar{\rho} & = 0.128 \pm 0.055\,,      &
    \bar{\eta} & = 0.375 \pm 0.060\,.
\end{aligned}
\end{equation*}
For the a-priori \gls{PDF}, we use uniform distributions for the BCL2008 \cite{Bourrely:2008za}
parameters. However, we construct a likelihood from the results of a recent study of the form factor
$f^{B\pi}_+(q^2)$ within \glspl{LCSR}. 
We now intend to draw random numbers from the posterior PDF using EOS' adaptive Metropolis-Hasting algorithm.
During its prerun phase, the algorithm adapts the chains' proposal
functions. We should demand at least $500$, and -- for a problem of this complexity -- maximally
$7500$ steps during the prerun phase; the adaption process should be executed
after every $500$ steps. For the final samples, we wish for a total of $10^4$, which we artifically decompose
into $10$ chunks with $1000$ samples each.\\

Our intentions translate to the following call to \client{eos-sample-mcmc}:
\commandlineexample{examples/sample-mcmc-btopi-ff}

We use the above call to \client{eos-sample-mcmc} in order to initialize a PMC run. We wish for
$4$ components per MC group, and to skip $20\%$ of the MCMC samples as part of the burn in.
Groups will be grouped based on an $R$-value threshold of $1.5$. For each update, $500$ samples
per mixture component shall be drawn, in order to produce $10^6$ samples in the final step.
Convergence shall be declared upon a standard deviation for the perplexity only, of $0.05$
over the last $4$ update steps. The call then reads:
\commandlineexample{examples/sample-pmc-btopi-ff}

\section{Plotting Random Parameter Samples}
\label{sec:usage:eos-plot}

Once random samples from the posterior PDF have been obtained, e.g. as described in
\refsec{usage:eos-sample-mcmc}, a visual inspection of the samples is the next step.
EOS provides scripts for this purpose, which plot histograms of either
a marginalized 1D (\texttt{eos-plot-1d}) or 2D (\texttt{eos-plot-2d}) probability density.
Both scripts presently detect the output format by inspection of the resepective HDF5 file
name. Files containing MCMC samples should be prefixed with \texttt{mcmc\_}, while
PMC sample files should be prefixed with \texttt{pmc\_monolithic\_}.

The \texttt{eos-plot-1d} produces a 1D histogram of the samples for one parameter.
It accepts the following arguments:
\begin{itemize}
    \item[] \texttt{HDF5FILE}\\[\medskipamount]
        The name of the HDF5 input file whose samples shall be plotted.

    \item[] \texttt{IDX}\\[\medskipamount]
        The numerical index for the parameter whose density function shall
        be plotted. \texttt{IDX} starts with zero

    \item[] \texttt{PDFFILE}\\[\medskipamount]
        The name of the PDF output file, into which the plot shall be saved.

    \item[] \texttt{--xmin XMIN}, \texttt{--xmax XMAX}\\[\medskipamount]
        When specified, limit the plot range to the interval \texttt{XMIN} to \texttt{XMAX}.
        The default values are taken from the description of the parameter within the
        HDF5 input file.

    \item[] \texttt{--kde [0|1]}\\[\medskipamount]
        When enabled, plots a univariate Kernel Density Estimate of the probability
        density based on the available samples.

    \item[] \texttt{--kde-bandwidth KDEBANDWIDTH}\\[\medskipamount]
        When specified, multiplies the automatically determined KDE bandwidth parameter
        with \texttt{KDEBANDWIDTH}.
\end{itemize}

The \texttt{eos-plot-2d} produces a 2D heatmap of the samples for two parameters.
It accepts the following arguments
\begin{itemize}
    \item[] \texttt{HDF5FILE}\\[\medskipamount]
        The name of the HDF5 input file whose samples shall be plotted.

    \item[] \texttt{XIDX}\\[\medskipamount]
        The numerical index for the parameter that shall be plotted on the
        $x$ acis. \texttt{XIDX} starts with zero

    \item[] \texttt{YIDX}\\[\medskipamount]
        The numerical index for the parameter that shall be plotted on the
        $y$ acis. \texttt{YIDX} starts with zero

    \item[] \texttt{PDFFILE}\\[\medskipamount]
        The name of the PDF output file, into which the plot shall be saved.

    \item[] \texttt{--xmin XMIN}, \texttt{--xmax XMAX}\\[\medskipamount]
        When specified, limit the plot range on the $x$ axis to the interval
        \texttt{XMIN} to \texttt{XMAX}. The default values are taken from the
        description of the parameter within the HDF5 input file.

    \item[] \texttt{--ymin YMIN}, \texttt{--ymax YMAX}\\[\medskipamount]
        When specified, limit the plot range on the $y$ axis to the interval
        \texttt{YMIN} to \texttt{YMAX}. The default values are taken from the
        description of the parameter within the HDF5 input file.
\end{itemize}

%\section{Bayesian Uncertainty Propagation using \textbf{\texttt{eos-propagate-uncertainty}}}
%\label{sec:usage:eos-propagate-uncertainty}
%
%Our intentions translate to the following call to \client{eos-sample-mcmc}:\todo{Does not work yet!}
%\commandlineexample{examples/propagate-uncertainty-b-to-pi-l-nu}
