\chapter{Installation}

The aim of this section is to assist you in the installation of EOS
from source. For the remainder of the section, we will assume that you
build and install EOS on a Linux based operating system, such as Debian
or Ubuntu.\footnote{%
    Other flavours of Linux will work as well, however, note that
    we will exclusively use package names as they appear in the Debian/Ubuntu
    \text{apt} databases.
}
Installation on MacOS X is known to work, but not guaranteed to work out
of the box.

\section{Installing the Dependencies}

The dependencies can be roughly categorized as either system software, or
scientific software.

\paragraph{System software} Installing EOS from source will require the following system
software to be pre-installed:
\begin{description}
    \item[\package{g++}] the GNU C++ compiler, in version 4.8.1 or higher,
    \item[\package{autoconf}] the GNU tool for creating configure scripts, in version 2.69 or higher,
    \item[\package{automake}] the GNU tool for creating makefiles, in version 1.14.1 or higher,
    \item[\package{libtool}] the GNU tool for generic library support, in version 2.4.2 or higher,
    \item[\package{pkg-config}] the freedesktop.org library helper, in version 0.28 or higher.
\end{description}
If you intend to use the Python \cite{Python} interface to the EOS library, you will need
to additionally install
\begin{description}
    \item[libboost-python-dev] the BOOST library for interfacing Python and C++.
\end{description}
We recommend that you install the above packages only via your system's software management system.

\paragraph{Scientific Software} Building and using the EOS core libraries requires in addition
the following scientific software to be pre-installed:
\begin{description}
    \item[\package{libgsl0-dev}] the GNU Scientific Library \cite{GSL}, in version 1.16 or higher,
    \item[\package{libhdf5-serial-dev}] the Hierarchical Data Format v5 library \cite{HDF5}, in version 1.8.11 or higher,
    \item[\package{libfftw3-dev}] the \dots,
    \item[\package{Minuit2}] the physics analysis tool for function minimization, in version 5.28.00 or higher.
\end{description}
Except for \package{Minuit2}, we recommend the installation of the above packages using your system's
software management system.\footnote{%
    There is presently a bug in the Debian/Ubuntu packages for \package{Minuit2}, which prevents
    linking.
}
For \package{Minuit2}, we recommend to install the package from source, and to
disable the automatic support for OpenMP. The installation can be done by
executing the following commands:
\begin{commandline}
mkdir /tmp/Minuit2
pushd /tmp/Minuit2
wget http://www.cern.ch/mathlibs/sw/5_28_00/Minuit2/Minuit2-5.28.00.tar.gz
tar zxf Minuit2-5.28.00.tar.gz
pushd Minuit2-5.28.00
./configure --prefix=/opt/pkgs/Minuit2-5.28.00 --disable-openmp
make all
sudo make install
popd
popd
rm -R /tmp/Minuit2
\end{commandline}

If you intend to use the \gls{PMC} sampling algorithm with EOS, you will need to install
\begin{description}
    \item[libpmc] a free implementation of said algorithm \cite{libpmc}, in version 1.01 or higher,
\end{description}
in addition to the core library dependencies. The \package{libpmc} package will -- in all likelihood -- not
be installable via your system's software management system. In addition, EOS requires some
modifications to libpmc's source code, in order to make it compatible with C++. We suggest
the following commands to install it:
\begin{commandline}
mkdir /tmp/libpmc
pushd /tmp/libpmc
wget http://www2.iap.fr/users/kilbinge/CosmoPMC/pmclib_v1.01.tar.gz
tar zxf pmclib_v1.01.tar.gz
pushd pmclib_v1.01
./waf configure --m64 --prefix=/opt/pkgs/pmclib-1.01
./waf
sudo ./waf install
sudo find /opt/pkgs/pmclib-1.01/include -name "*.h" \
    -exec sed -i \
    -e 's/#include "errorlist\.h"/#include <pmctools\/errorlist.h>/' \
    -e 's/#include "io\.h"/#include <pmctools\/io.h>/' \
    -e 's/#include "mvdens\.h"/#include <pmctools\/mvdens.h>/' \
    -e 's/#include "maths\.h"/#include <pmctools\/maths.h>/' \
    -e 's/#include "maths_base\.h"/#include <pmctools\/maths_base.h>/' \
    {} \;
sudo sed -i \
    -e 's/^double fmin(double/\/\/&/' \
    -e 's/^double fmax(double/\/\/&/' \
    /opt/pkgs/pmclib-1.01/include/pmctools/maths.h
popd
popd
rm -R /tmp/libpmc
\end{commandline}


\section{Installing EOS}

The most recent version of EOS is contained in the public GIT \cite{GIT}
repository at \url{http://github.com/eos/eos}.  In order to download it for the
first time, create a new local clone of said repository via:
%
\begin{commandline}
git clone \
    -o eos \
    -b master \
    https://github.com/eos/eos.git \
    eos
\end{commandline}


As a first step, you need to create all the necessary build scripts via:
%
\begin{commandline}
cd eos
./autogen.bash
\end{commandline}
%
Next, you configure the build scripts using:
%
\begin{commandline}
./configure \
    --prefix=/opt/pkgs/eos \
    --enable-python \
    --with-minuit2=/opt/pkgs/Minuit2-5.28.00 \
    --with-pmc=no
    # alternative 1: --disable-python
    # alternative 2: --with-minuit2=root
    # alternative 3: --with-pmc=/opt/pkgs/pmclib-1.01
\end{commandline}
%
In the above, three alternatives apply:
\begin{enumerate}
    \item Building the EOS-Python interface is purely optional. In order to disable building
    this interface, replace \texttt{--enable-python} with \texttt{--disable-python}.
    \item If you have installed ROOT on your system, you can use ROOT's internal copy of Minuit2.
    In such a case, issue \texttt{--with-minuit2=root} to the configure script.
    \item Building the EOS-PMC support is purely optional. In order to enable building this
    support, issue \texttt{--with-pmc} with the appropriate installation path.
\end{enumerate}
If the \texttt{configure} script finds any problems with your system, it will complain loudly.\\

After successful configuration, you can build and install EOS using:
%
\begin{commandline}
make all
sudo make install
\end{commandline}
%
Moreover, we urgently recommend to also run the complete test suite by executing:
%
\begin{commandline}
make check
\end{commandline}
%
within the source directory. Please contact the authors in the case that any
test failures should occur.


In order to be able to use the EOS clients from the command line, you will need to
set up some environment variables. For the Bash, which is the default Debian/Ubuntu
shell this can be achieved by adding the lines
\begin{commandline}
export PATH+=":/opt/pkgs/eos/bin"
export PYTHONPATH+=":/opt/pkgs/eos/lib/python2.7/site-packages"
\end{commandline}
to you \texttt{.bashrc} file. In the above, the last line is optional and should only be
added if you built EOS with Python support. Note that \texttt{python2.7} should be
replaced by the apppropriate Python version against which EOS was built.


In order to build you own programs that use the EOS libraries,
add
\begin{commandline}
CXXFLAGS+=" -I/opt/pkgs/eos/include"
LDFLAGS+=" -L/opt/pkgs/eos/lib"
\end{commandline}
to your makefile.

\paragraph{Python 3} If you intend to build the EOS-Python interface using Python 3, there will be
additional steps required on Debian/Ubuntu. You will need to pass the environment variables
\texttt{PYTHON} and \texttt{BOOST\_PYTHON\_SUFFIX} to the configure script, e.g. like this:
\begin{commandline}
PYTHON=python3 BOOST_PYTHON_SUFFIX=-py34 ./configure \
    ...
\end{commandline}
where the dots indicate all the options passed to \texttt{configure} on the commandline.
Note that is is a bad idea to simultaneously install the EOS-Python interface for both
Python 2.x and Python 3.x, since the \texttt{PYTHONPATH} environment variable is used
by both versions and there is no versioning support for Python modules written in C++.
